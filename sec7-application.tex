\section{Application of the Science Platform inside the Project}\label{application-of-the-science-platform-inside-the-project}

The Science Platform is not only for use by scientist end users through the Data Access Centers but it also forms an integral part of the toolchain within the Project. A phased release of capabilities focuses on the needs of the primary target audience for each major version. This approach allows us to deliver features of the platform to project staff to assist them in contruction work while also providing an early user base whose feedback will allow us to refine our design. 

\subsection{Data Management Teams}\label{developer-support}

The earliest releases of the Science Platform are aimed at Data Management teams. The capabilities supporting software development include interfactively troubleshooting code and analysing metrics through the Notebook Aspect. The capabilities supporting algorithmisc development include visualising the results of large processing runs of precursor and simulated data through the Portal Aspect. 

\subsection{Commissioning Team}\label{commissioning}

Intermediary releases of the Science Platform are aimed at the commissioning team. The capabilities supporting commissioning include the interfaces needed to allow Notebooks to cross-correlate early ComCam data with telemetry captured in the Engineering Facilities Database and a deployment of the Science Platform on the Commissioning Cluster.

\subsection{Science Validation Team} 

Mature releases of the Science Platform are aimed at the Science Verification Team which is expected to involve a mixture of Project and Science Collaboration members. Science Validation requires many of the featurs of the operational Science Platform since it involves detailed interactive data analysis and generating and visualising statistics from large data runs.

\subsection{Observatory Operations}\label{observatory-operations}

In the Operations era, all of the above workflows will continue to be in heavy use by Observatory staff: Software support will continue to characterise releases using the notebooks and reporting tools developed in Construction; facility checkout after engineering periods at the telescope will re-use many of the processes developed in commissioning; and facility scientists and calibration specialists will use workflows similar to those developed druing science validation to enhance our understanding of the instrument. This will ensure that all Science Platform Aspects made available to the community remain vigorous and relevant.
